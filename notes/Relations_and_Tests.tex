\documentclass[11pt]{article}
\usepackage[utf8]{inputenc} 
\usepackage[top=1cm, bottom=2.0cm, left=1.0cm, right=1.5cm]{geometry}
\usepackage{graphicx}
\usepackage{array}
\usepackage{bbold}
\usepackage{amsmath,amssymb}
\usepackage{caption}
\usepackage{hyperref}
\usepackage{cite}
\usepackage{algpseudocode,algorithm,algorithmicx}
\usepackage{empheq}
\usepackage{lscape}
\newcommand*\widefbox[1]{\fbox{\hspace{2em}#1\hspace{2em}}}
\newcommand{\vv}[1]{\mathbf{#1}}
\newcommand{\ii}[0]{\mathrm{i}}
\newcommand{\dd}[0]{\mathrm{d}}
\DeclareMathOperator\arctanh{arctanh}
% define citation style
\bibliographystyle{unsrt}
\title{Test Results in the Context of Quantum/Casimir Friction}
\author{Marty Oelschläger}
\begin{document}
\maketitle
%\tableofcontents
\section{Response Functions}
All response functions ($\underline{G}$, $\underline{\alpha}$, $\epsilon$, etc.) have to fulfill a certain set of easily testable relations. First, they have to fulfill reality in time domain and thus the crossing relation in frequency domain
\begin{align}
\underline{G}(-\omega,\vv r, \vv r') &= \underline{G}^\dagger(\omega,\vv r, \vv r')\,,  \\
\underline{\alpha}(-\omega) &= \underline{\alpha}^*(\omega) \,,\\
\epsilon(-\omega) &= \epsilon^*(\omega)  \,,\\
r(-\omega) &= r^*(\omega)  \,.\\
\end{align}
Furthermore, in most cases one can rely on reciprocity of the Green's tensor
\begin{align}
\underline{G}(\omega,-\vv k,z) &= \underline{G}^\intercal(\omega,\vv k, z)\,.
\end{align}
In order to test the Green's tensor's reality, one can use that
\begin{align}
\underline{G}_\Im(\omega,\vv k, z) =\frac{\underline{G}(\omega,\vv k, z) - \underline{G}^\dagger(\omega,\vv k, z)}{2\ii}=
\frac{\underline{G}^\dagger(-\omega,\vv k, z) - \underline{G}(-\omega,\vv k, z)}{2\ii}= -\underline{G}_\Im(-\omega,\vv k, z)\,,\\
\underline{G}_\Re(\omega,\vv k, z) =\frac{\underline{G}(\omega,\vv k, z) + \underline{G}^\dagger(\omega,\vv k, z)}{2}=
\frac{\underline{G}^\dagger(-\omega,\vv k, z) + \underline{G}(-\omega,\vv k, z)}{2}= \underline{G}_\Re(-\omega,\vv k, z)
\,.
\end{align}
However, it is more feasible to compare the integrated quantities 
\begin{align}
  \int\frac{\dd^2\vv k}{(2\pi)^2}
\underline{G}_\Im(\omega,\vv k, z) = 
-\int\frac{\dd^2\vv k}{(2\pi)^2} \underline{G}_\Im(-\omega,\vv k, z)\,,\\
\end{align}
\section{Power Spectrum and Friction}
Another relevant quantity in the context of the noncontact friction is the power spectrum $\underline{S}$ (see for example \cite{intravaia2016}). Here, we find the testable relation
\begin{align}
\underline{S}^\dagger(\omega) &= \underline{S}(\omega)\,.
\end{align}
Moreover, one should be able to retrieve the static case
\begin{align}
\lim_{v\to0}\underline{S}(\omega) &\sim \frac{\hbar}{\pi}\theta(\omega)\underline{\alpha}_\Im(\omega)\,.
\end{align}
Additionally, one can test the integration of $\underline{\alpha}$ in the first order in $\alpha_0$.
\begin{align}
\int_0^{\omega_\mathrm{cut}}\dd\omega\,\underline{\alpha}_\Im(\omega)
\sim
\begin{cases}
\mathbb{1}\frac{\pi}{2}\alpha_0\omega_a
\quad  & \text{for}\quad \omega_\mathrm{cut}\gg \omega_a
\\
\alpha_0^2
\int_0^{\omega_\mathrm{cut}}\dd\omega\,
\int\frac{\dd^2\vv k}{(2\pi)^2}
\underline{G}_\Im(\vv k, z_a,\omega + k_xv)
\quad  &\text{for}\quad \omega_\mathrm{cut}\ll \omega_a
\end{cases}
\end{align}
\section{Test Case: Free Space}
In free space for temperatures $k_\mathrm{B}T\gg \hbar\omega_a$ one finds (see \texttt{VacuumGreen.pdf})
\begin{align}
F &= -\frac{v}{c} \frac{k_\mathrm{B}T \alpha_0}{6\pi\epsilon_0} \frac{\omega_a^4}{c^4}
\end{align}
Furthermore, one can test the individual Green's tensor integrations via the relations given in the \texttt{VacuumGreen.pdf}. With respect to the polarizability one finds
\begin{align}
\int_0^{\omega_\mathrm{cut}}\dd\omega\,\underline{\alpha}_\Im(\omega)
\sim
\begin{cases}
\mathbb{1}\frac{\pi}{2}\alpha_0\omega_a
\quad  & \text{for}\quad \omega_\mathrm{cut}\gg \omega_a
\\
\alpha_0^2
  \frac{\omega_\mathrm{cut}^4}{8\pi\epsilon_0}\mathrm{diag}\left[
  \frac{ c }{3 \left(c^2-v^2\right)^2}
    ,\,
    \frac{ c  \left(c^2+v^2\right)}{3 \left(c^2-v^2\right)^3}
  ,\,
\frac{ c  \left(c^2+v^2\right)}{3 \left(c^2-v^2\right)^3}  
\right]
\stackrel{v\ll c}\approx
\mathbb{1}\alpha_0^2 \frac{\omega_\mathrm{cut}^4}{24\pi\epsilon_0 c^3} 
\quad  &\text{for}\quad \omega_\mathrm{cut}\ll \omega_a
\end{cases}
\end{align}
\bibliography{lib}
\end{document}
